\subsection{The Bochner Vanishing Theorem}

Let us consider a dimension \(n > 1\).

Consider what happens if we were to start with \(M = S^n\), and we then mod out by the action of a
discrete isometry sub-group \(G\) of \(M\). Can we obtain a manifold \(N = M / G\) such that \(N\)
has \(N\) has a non-trivial cohomology group \(H^1(N, \mathbb R)\)? Recall that such an action is easily 
possible for actions on \(\mathbb R^n\), i.e. the torus \(T^n\) has \(H^1(T^n, \mathbb R) = \mathbb R^n\).
Our question is related to what happens if we start with \(M = S^n\) (note that it is also simply-connected).

The Bochner Vanishing Theorem tells us that it is impossible to use discrete isometry groups to mod out \(M = S^n\)
to make a new manifold with a non-vanishing de-Rham cohomology group \(H^1\). The key to Bochner's method
has nothing to do directly with the topology of \(S^n\). Instead we are lead to the result by the 
curvature of \(S^n\). In fact, Bochner's theorem applies to a much wider range of examples where we know
that the resulting manifold \(N\) has positive curvature. 

His idea is to find special priveleged class of one-forms (i.e. the harmonic one-forms) for each cohomology class,
and these one-forms have strong interactions with the geometry of \(N\), i.e. the curvature of \(N\). 

Bochner's example comes in several pieces:
\begin{enumerate}
\item Show that there exists harmonic forms in every cohomology class.
\item Prove that there is enough regularity for these forms to use their Euler-Lagrange equations.
\item Use their relationship with the curvature of \(N\) to show that in cases of positive curvature,
these one-forms must vanish. 
\end{enumerate}
